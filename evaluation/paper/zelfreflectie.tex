\section{Competencies}
\textbf{Theoretical and mathematical foundations of computer science}: I wanted to expand my knowledge of the theoretical and mathematical foundations of computer science. I found that I have improved this competency. More specifically, I learned about algebraic effect handlers and formal type systems. However, my thirst for more knowledge has only increased since starting the honoursproject. I want to learn more about type systems and algebraic effect handlers, especially about the theoretical foundations. \\
\\
\textbf{Oral communication}: I learned a lot about this competency. I was able to discuss decisions to be made in the project with other researchers. In the weekly meetings, each researcher had to talk about the progress made and the issues encountered during the previous week. I feel like I was more shy towards the beginning of the honoursproject. This has gotten a lot better during the honoursproject. However, I feel like I cannot always express my thoughts in the correct scientific terms. That is definitly a working point for the future. Most communication happened in English. It was really interesting and fun to communicate professionally in English. I didn't think that improving my English communication skills was a competency I was going to learn. However, my communication skills did improve during the honoursproject. A skill that I wanted to improve was given presentations for a larger audience. However, this is something that does not really occur during the honoursproject. \\
\\
\textbf{Written communication}: I improved this competency less than I expected. It also improved in a different way than I expected. I expected to improve my competency in writing reports. However, I didn't write many reports for this honoursproject. I did write portions of the paper, which improved my academic writing competency. This is a competency that still needs to be improved a lot. More specifically, my academic writing skills need to be improved. \\
\\
\textbf{Asking for help}: As explained before, this is one of the most important competencies that I learned. I didn't think I would learn this competency, since I never realised that this was something that needed improvement. In a research project, it is important to understand the material, but there is also a deadline. That means that without asking for help, things are bound to go awry.  This is something I learned during the honoursproject. However, I do think that this is something that can and needs improvement. Sometimes it occurs that people explain a new concept to me, but I don't fully understand it yet. In those circumstances, I would like to ask for more explanations, however I'm not always sure how to ask for help. I also think that sometimes I assume too quickly that I understand something when it would have been better to ask more questions. These are definitely issues that have a high priority for improvement.\\
\\
\textbf{Time management}: In my reflection I explained that I improved my time management skills. I find this to be an important competency to be learned. Due to the huge workload that I took, which wasn't solely due to the honoursproject, I learned how to efficiently manage my time. I do think that there are still many improvements possible. More specifically, I'm not good at estimating how long a task will take me to execute. I often underestimate or overestimate the amount of time required for certain tasks. This happens mostly with tasks that take less than a week to complete.  \\
\\
\textbf{Interdisciplinary interest}: This competency is not directly linked to my honoursproject, but is more a consequence of the honoursprogramme. Nevertheless, I still find this to be important enough to include in this report. My honoursproject is focused on research and also happens to be within my own discipline. Due to this, the interdisciplinary character of my honoursproject is small.  Due to being part of the honoursprogramme, I also attended intervision moments and honourscommunity meetings. In these meetings I met a lot of new people. Due to these events I learned how different people look towards honoursprogrammes and interdisciplinary work. This is something that broadened my view and is definitely an aspect which I find important. \\
\\
\textbf{Finding my limit}:  As explained above, I improved my time management skills. Another related competency that I improved is how to find my own limits, what I can and cannot do. This is another competency I learned due to the high workload I had. Using the honoursproject I found the bounderies of the amount of work I can do. However, I do think I'm sometimes still overzealous. \\
\\
\textbf{Creating a hypothesis}: This is a competency I wanted to learn. I feel like this is something that improved over the course of the honoursproject. However I still find it difficult to create a hypothesis. An illustration of this is my masterthesis topic. I knew several aspects about my topic. I wanted it to be research related and it should be related to my honoursproject, type systems. However, it was quite difficult to get concrete ideas myself. Partly I believe this comes from not being knowledgeable enough about the field, which is normal since I'm a masterstudent. But partly I also believe that I also need to learn how to look for potential research ideas. 