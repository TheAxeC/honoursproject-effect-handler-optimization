\subsection{Looking back}
I was able to accomplish more than I expected in the honoursproject. I expected that I could only implement one optimization and that by then I would be done. Multiple optimizations were implemented. In this regard, I underestimated the amount of work I could do in the honoursproject. \\
\\
However, It took longer to orientate myself in the landscape of algebraic effect handlers than I expected. I found that there is a big difference between getting familiar enough with algebraic effects and handlers so I could use them within programs and getting familiar enough with algebraic effects and handlers so I could do research within that field. More specifically, I could write small programs that use algebraic effects and handlers quite fast, but it took a lot longer to be able to do the optimizations. 

\subsection{Where the goals reached}
In the motivation letter, multiple steps were given that needed to be accomplished.
\begin{enumerate}
\item Literature study of Eff and existing optimizations \label{literature}
\item Designing new optimizations \label{design}
\item implementation \label{implementation}
\item evaluation through benchmarks \label{evaluation}
\item formal proof of the optimizations \label{proof}
\end{enumerate}
These steps, with the exclusion of step~\ref{proof}, were accomplished. In the group, it was decided that it was best for me to continue working on the benchmarks and writing the paper instead of writing formal proofs. \\
\\
Step~\ref{literature} was the first task that was accomplished. This was essential for me to be able to contribute to the research project. Afterwards step~\ref{design} and step~\ref{implementation} were done in parallel. Finally step~\ref{evaluation} and working on the paper were done. 

\subsection{Value of the Honoursprogramme}
Other goals that are important for the honoursprogramme concern my own contributions and the knowledge and experience that were gained. During the honoursproject, I was partly treated as an honoursstudent and partly as a researcher. I was partly treated as an honourstudent since I did have regular meetings with professor Schrijvers. But I was also treated as a researcher as I did have a voice during meetings for the project. This meant that I was able to make contributions and state my opinion about decisions before they are made. Ofcourse this came with the responsibility that I carried my own weight.\\
\\
I would say there were a lot of values to the honoursproject. Some I have already mentioned. I got introduced to being a researcher. The honoursproject was quite a challenge, which I definitely found a value of the honoursproject.  I believe that because of these values, my thirst for more knowledge also increased. I want to delve deeper into type-\&-effect systems, learn more about the theoretical foundations. 

\subsection{Conclusion}
To conclude, something that definitely needs to be said is that I'm proud of what I've accomplished during this honoursproject. Choosing to participate in the honoursprogramme and doing my honoursproject with professor Schrijvers were the best choices I could have made. The honoursproject taught me a lot about research, programming language theory and myself. 
